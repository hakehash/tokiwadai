\documentclass[11pt]{jsarticle}
\usepackage{type1ec}
\usepackage[OT2,T1]{fontenc}
\usepackage[russian,english,japanese]{babel}
\usepackage{multicolpar}
\begin{document}
\subsection*{常盤台ロシア語学校(仮称) 不定期通信6}
\begin{flushright}
  2018-07-20
\end{flushright}
\selectlanguage{russian}
特に伝達事項はありませんが, Urok 6と7で形容詞について学習したので単語とフレーズをいくつか紹介しておきます.
\subsubsection*{今週の単語}
\begin{table}[ht]
  \begin{tabular}{ll}
    v\'eqna{j1} merzlot\'a & 永久凍土 \\
    {e1}kolog\'iqeski{i0}\ sled & エコロジカル・フットプリント \\
    isk\'usstvenna{j1}\ ne{i0}r\'onna{j1} set{p1} & ニューラルネットワーク \\
    isk\'usstvenny{i0}\ [形] & 人工の \\
    isk\'usstvenny{i0}\ sp\'utnik & 人工衛星 \\
    kosm\'inqeski{i0}\ l\'ift & 軌道エレベータ \\
    obyknov\'enna{j1}\ r\'yba-s\'abl{j1} & 太刀魚 \\
    obyknov\'enny{i0}\ [形] & ありふれた, 普通の, 平凡な \\
    olimp\'i{i0}skie \'igry & オリンピック大会 \\
    s\'abl{j1} & サーベル \\
    serv\'al & サーバル
  \end{tabular}
\end{table}
\subsubsection*{先々週と先週のフレーズ}
\noindent
"--* Po{i0}d{e0}mte na zv{e0}zdny{i0} festiv\'al{p1} v subb\'otu. \\
"--* Izvin\'ite, {j1} \'oqen{p1} z\'an{j1}t v \'\cyrerev ti vyhodn\'ye, po\'\cyrerev tomu {j1} ne mog\'u. \\
"--* 土曜日に七夕まつりへ行きましょう.\\
"--* すみませんが, 私は今週末とても忙しいので行けません.\\
\\
"--* Kogd\'a vy po\'edete v universit\'et z\'avtra?\\
"--* V 10 qas\'ov.\\
"--* A u vas est{p1} svob\'odnoe vr\'em{j1} z\'avtra v 15 qas\'ov?\\
"--* 明日はいつ大学へ来ますか?\\
"--* 10時に.\\
"--* では明日の15時はお暇ですか?\\
\\
単語: zv{e0}zdny{i0} festiv\'al{p1} 七夕まつり, subb\'ota 土曜日, vyhodn\'ye 週末, po\'ehat{p1} [完] (乗り物で)行く
\end{document}
