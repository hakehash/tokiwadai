\documentclass[11pt]{jsarticle}
\usepackage{type1ec}
\usepackage[OT2,T1]{fontenc}
\usepackage[russian,english,japanese]{babel}
\begin{document}
\subsection*{常盤台ロシア語学校(仮称) 授業案内その2}
\begin{flushright}
  2018-05-29
\end{flushright}
(文字と発音が予定通り終われば)次回(2018-06-01)から例文を用いた練習を始めます.
テキストはさしあたって 東一夫・東多喜子(2003)『標準ロシア語入門』白水社 pp.18-21を使いますが, 
他に良いものがあれば遠慮なく提案してください.\\
発音の確認・練習, 口頭での露文和訳, 和文露訳を行いますので, 
可能ならば予習をしておく(付属のCDで発音を確認しておくなど)ことが望ましいです.\\
それから, 前回書き忘れましたが, ノートを用意されるのであれば, 英語用の4本線のものが便利です.
\selectlanguage{russian}
\subsubsection*{}
\subsubsection*{今週の単語}
\begin{table}[ht]
  \begin{tabular}{ll}
    b\'ela{j1} dosk\'a & ホワイトボード \\
    dosk\'a & 板, 黒板 \\
    kr\'asna{j1} pl\'owad{p1} & 赤の広場 \\
    sneg & 雪 \\
    snoub\'ord & スノーボード
  \end{tabular}
\end{table}
\subsubsection*{先週のフレーズ}
\noindent
\--- Z\'avtra z\'avtra ne seg\'odn{j1}, tak lent\'\cyrya{i0} govor\'\cyrya t. (germ\'anska{j1} posl\'ovica)\\
\--- 明日, 明日, 今日でなく. 怠け者はそう言う. (ドイツの諺)\\
\\
単語: seg\'odn{j1} 今日, tak そのように, lent\'\cyrya{i0} 怠け者, govor\'it{p1} 話す[不完]
\end{document}
