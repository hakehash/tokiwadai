\documentclass[a4j,11pt]{jsarticle}
\usepackage{type1ec}
\usepackage[OT2,T1]{fontenc}
\usepackage[russian,english,japanese]{babel}
\usepackage{multicolpar}
\usepackage{okumacro}
\begin{document}
\subsection*{常盤台ロシア語学校(仮称) 単語テスト2}
\begin{flushright}
  2018-06-20
\end{flushright}
\selectlanguage{russian}
\subsubsection*{単語テスト 各2点\\(単語にはウダレーニエ(アクセントの記号)をふること)}
\begin{multicolpar}{2}
\underline{\phantom{Моя}} mat{p1}\ \---\ domohoz\'\cyrya{i0}ka, a \underline{\phantom{мой}} otec\ \---\ pension\'er. A \underline{\phantom{его}} d\'eduxka \---\ rab\'oqi{i0}.

私の母は主婦で, 私の父は年金生活者です. 一方, 彼の祖父は労働者です.\\
\end{multicolpar}
\begin{multicolpar}{2}
\underline{\phantom{Наша}} stran\'a i \underline{\phantom{наше}} m\'ore.

私たちの国と私たちの海
\end{multicolpar}
\begin{multicolpar}{2}
\underline{\phantom{Их}} p\'esn{j1}.

彼らの歌
\end{multicolpar}
\begin{multicolpar}{2}
\--- \underline{\phantom{Чей}} \'\cyrerev to zont?\\
\--- \'\CYREREV to \underline{\phantom{твой}} zont.

\noindent
\--- これは誰の傘ですか?\\
\--- これは君の傘だよ.\\
\end{multicolpar}
\begin{multicolpar}{2}
Mne nr\'av{j1}{t}s{j1} \underline{\phantom{ваши}} slov\'a.

私はあなたの言葉が気に入りました.
\end{multicolpar}
\begin{multicolpar}{2}
\--- Kto \underline{\phantom{этот}} qelov\'ek?\\
\--- \'\CYREREV to \underline{\phantom{её}} mu{zh}.

\noindent
\--- この人は誰ですか?\\
\--- これは彼女の夫です.
\end{multicolpar}
\subsubsection*{先々週と先週のフレーズ}
\begin{multicolpar}{2}
\--- Kogd\'a vy vsta{e0}te?\\
\--- {J1} vsta\'\cyryu\ v d\'es{j1}t{p1} qas\'ov.\\
\--- Pr\'avda?\\

\noindent
\--- あなたはいつ起きますか?\\
\--- 私は10時に起きます.\\
\--- 本当に?\\
\end{multicolpar}
\begin{multicolpar}{2}
\--- Qto u vas bol\'it?\\
\--- U men\'\cyrya\ bol\'it golov\'a.\\
\--- {Zh}\'alko.

\noindent
\--- あなたはどこが痛むのですか?\\
\--- 私は頭が痛いです.\\
\--- かわいそうに.
\end{multicolpar}
単語: vstav\'at{p1} 起き上がる, d\'es{j1}t{p1} 10, qas 時, golov\'a 頭
\subsubsection*{vaxの用例 (I. ツルゲーネフ \guillemotleft As{j1}\guillemotright\ 16節、17節より 訳: 二葉亭四迷(1896)『片恋』)}
\noindent
\--- V\'axa... \--- proxept\'ala ona qut{p1} sl\'yxno.\\
%\ruby{戦}{わなゝ}くやうな、\ruby{断続}{ちぎれちぎれ}の\ruby{太息}{ためいき}が\ruby{聞}{き}こえて、
「\ruby{死}{し}んでも\ruby{可}{い}いわ…」とアーシヤは\ruby{云}{い}つたが、\ruby{聞取}{きゝと}れるか\ruby{聞取}{きゝと}れぬ\ruby{程}{ほど}の\ruby{小聲}{こごえ}であつた。\\
\flqq{}V\'axa...\frqq \--- sl\'yxals{j1} mne e{e0} x{e0}pot.\\
「\ruby{死}{し}んでも\ruby{可}{い}いわ…」と\ruby{幽}{かすか}に\ruby{云}{い}つた\ruby{其聲}{そのこゑ}が、\ruby{未}{ま}だ\ruby{耳元}{みみもと}に\ruby{聞}{き}こえる
\end{document}
