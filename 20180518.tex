\documentclass[11pt]{jsarticle}
\usepackage{type1ec}
\usepackage[OT2,T1]{fontenc}
\usepackage[russian,english,japanese]{babel}
\begin{document}
\subsection*{シラバスに代えて \--- 常盤台ロシア語学校(仮称) 開講案内}
\subsubsection*{開講日時}
\noindent
毎週火曜日09:00からと毎週金曜日17:00からの週2回, 各回1時間前後を予定しています.\\
\selectlanguage{russian}
Ur\'ok 1(第一講)は文字と発音, 2018-05-22 09:00より始めます. (場所: 構造研院生室)
\subsubsection*{教科書}
今のところ
\begin{itemize}
  \item 桑野隆(2012)『初級ロシア語20課』白水社
  \item 東一夫・東多喜子(2003)『標準ロシア語入門』白水社
\end{itemize}
が候補です.\\
\selectlanguage{english}
購入は強制しませんが, いずれもCD付なので, 持っていると発音の練習に便利です.
\selectlanguage{russian}
\subsubsection*{辞書}
『博友社ロシア語辞典』が理想的ですが, やや値が張ります.\\
より安価な辞書の中では白水社の『パスポート初級露和辞典』が親切です.
\subsubsection*{今週の単語}
\begin{table}[ht]
  \begin{tabular}{ll}
    aspir\'ant & 大学院生\\
    aspir\'antka & 大学院生(女子)\\
    doc\'ent & 准教授\\
    prof\'essor & 教授\\
    storo{zh}ev\'a{j1} sob\'aka & 番犬\\
    udar\'enie & ウダレーニエ(アクセント)
  \end{tabular}
\end{table}
\subsubsection*{先週のフレーズ}
\noindent
\--- Vy pokup\'ali uq\'ebnik r\'usskogo {j1}zyk\'a?\\
\--- Net, {j1} b\'udu kup\'it{p1} eg\'o z\'avtra.\\
\\
\--- あなたはロシア語の教科書を買いましたか?\\
\--- いいえ, 明日買います.\\
\\
単語:\\vy あなた(は), z\'avtra 明日, kup\'it{p1} 買う[不完], r\'usski{i0} {j1}z\'yk ロシア語, uq\'ebnik 教科書
\end{document}
