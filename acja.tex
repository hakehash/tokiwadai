В небольшой комнате, куда я вошел, было довольно темно, и я не тотчас увидел Асю. Закутанная в длинную шаль, она сидела на стуле возле окна, отвернув и почти спрятав голову, как испуганная птичка. Она дышала быстро и вся дрожала. Мне стало несказанно жалко ее. Я подошел к ней. Она еще больше отвернула голову…

— Анна Николаевна, — сказал я.

Она вдруг вся выпрямилась, хотела взглянуть на меня — и не могла. Я схватил ее руку, она была холодна и лежала, как мертвая, на моей ладони.

— Я желала… — начала Ася, стараясь улыбнуться, но ее бледные губы не слушались ее, — я хотела… Нет, не могу, — проговорила она и умолкла. Действительно, голос ее прерывался на каждом слове.

Я сел подле нее.

— Анна Николаевна, — повторил я и тоже не мог ничего прибавить.

Настало молчание. Я продолжал держать ее руку и глядел на нее. Она по-прежнему вся сжималась, дышала с трудом и тихонько покусывала нижнюю губу, чтобы не заплакать, чтобы удержать накипавшие слезы… Я глядел на нее; было что-то трогательно-беспомощное в ее робкой неподвижности: точно она от усталости едва добралась до стула и так и упала на него. Сердце во мне растаяло…

— Ася, — сказал я чуть слышно…

Она медленно подняла на меня свои глаза… О, взгляд женщины, которая полюбила, — кто тебя опишет? Они молили, эти глаза, они доверялись, вопрошали, отдавались… Я не мог противиться их обаянию. Тонкий огонь пробежал по мне жгучими иглами, я нагнулся и приник к ее руке…

Послышался трепетный звук, похожий на прерывистый вздох, и я почувствовал на моих волосах прикосновение слабой, как лист дрожавшей руки. Я поднял голову и увидел ее лицо. Как оно вдруг преобразилось! Выражение страха исчезло с него, взор ушел куда-то далеко и увлекал меня за собою, губы слегка раскрылись, лоб побледнел как мрамор, и кудри отодвинулись назад, как будто ветер их откинул. Я забыл все, я потянул ее к себе — покорно повиновалась ее рука, все тело ее повлеклось вслед за рукою, шаль покатилась с плеч, и голова ее тихо легла на мою грудь, легла под мои загоревшиеся губы…

— Ваша… — прошептала она чуть слышно.

Уже руки мои скользили вокруг ее стана… Но вдруг воспоминание о Гагине, как молния, меня озарило.

— Что мы делаем!.. — воскликнул я и судорожно отодвинулся назад. — Ваш брат… ведь он все знает… Он знает, что я вижусь с вами.

Ася опустилась на стул.

— Да, — продолжал я, вставая и отходя в другой угол комнаты. — Ваш брат все знает… Я должен был ему все сказать.

— Должны? — проговорила она невнятно. Она, видимо, не могла еще прийти в себя и плохо меня понимала.

— Да, да, — повторил я с каким-то ожесточением, — и в этом вы одни виноваты, вы одни. Зачем вы сами выдали вашу тайну? Кто заставлял вас все высказать вашему брату? Он сегодня был сам у меня и передал мне ваш разговор с ним. — Я старался не глядеть на Асю и ходил большими шагами по комнате. — Теперь все пропало, все, все.

Ася поднялась было со стула.

— Останьтесь, — воскликнул я, — останьтесь, прошу вас. Вы имеете дело с честным человеком, — да, с честным человеком. Но, ради бога, что взволновало вас? Разве вы заметили во мне какую перемену? А я не мог скрываться перед вашим братом, когда он пришел сегодня ко мне.

«Что я такое говорю?» — думал я про себя, и мысль, что я безнравственный обманщик, что Гагин знает о нашем свидании, что все искажено, обнаружено, — так и звенела у меня в голове.

— Я не звала брата, — послышался испуганный шепот Аси, — он пришел сам.

— Посмотрите же, что вы наделали, — продолжал я. — Теперь вы хотите уехать…

— Да, я должна уехать, — так же тихо проговорила она, — я и попросила вас сюда для того только, чтобы проститься с вами.

— И вы думаете, — возразил я, — мне будет легко с вами расстаться?

— Но зачем же вы сказали брату? — с недоумением повторила Ася.

— Я вам говорю — я не мог поступить иначе. Если бы вы сами не выдали себя…

— Я заперлась в моей комнате, — возразила она простодушно, — — я не знала, что у моей хозяйки был другой ключ…

Это невинное извинение, в ее устах, в такую минуту — меня тогда чуть не рассердило… а теперь я без умиления не могу его вспомнить. Бедное, честное, искреннее дитя!

— И вот теперь все кончено! — начал я снова. — Все. Теперь нам должно расстаться. — Я украдкой взглянул на Асю… лицо ее быстро краснело. Ей, я это чувствовал, и стыдно становилось и страшно. Я сам ходил и говорил как в лихорадке. — Вы не дали развиться чувству, которое начинало созревать, вы сами разорвали нашу связь, вы не имели ко мне доверия, вы усомнились во мне…

Пока я говорил, Ася все больше и больше наклонялась вперед — и вдруг упала на колени, уронила голову на руки и зарыдала. Я подбежал к ней, пытался поднять ее, но она мне не давалась. Я не выношу женских слез: при виде их я тотчас теряюсь.

— Анна Николаевна, Ася, — твердил я, — пожалуйста, умоляю вас, ради бога, перестаньте… — Я снова взял ее за руку…

Но, к величайшему моему изумлению, она вдруг вскочила — с быстротою молнии бросилась к двери и исчезла…

Когда несколько минут спустя фрау Луизе вошла в комнату — я все еще стоял по самой середине ее, уж точно как громом пораженный. Я не понимал, как могло это свидание так быстро, так глупо кончиться — кончиться, когда я и сотой доли не сказал того, что хотел, что должен был сказать, когда я еще сам не знал, чем оно разрешиться…

— Фрейлейн ушла? — спросила меня фрау Луизе, приподняв свои желтые брови до самой накладки.

Я посмотрел на нее как дурак — и вышел вон.
