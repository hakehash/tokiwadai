\documentclass[11pt]{jsarticle}
\usepackage{type1ec}
\usepackage[OT2,T1]{fontenc}
\usepackage[russian,english,japanese]{babel}
\usepackage{multicol}
\usepackage{okumacro}
\begin{document}
\subsection*{常盤台ロシア語学校(仮称) 不定期通信4}
\begin{flushright}
  2018-06-08
\end{flushright}
\subsubsection*{語学学習に便利なサイト・アプリ}
\begin{itemize}
\item{Webサイト}\\
東京外国語大学言語モジュール|ロシア語\\
http://www.coelang.tufs.ac.jp/mt/ru/\\
ロシア語独習コンテンツ - 大阪大学世界言語eラーニングサーバ\\
http://el.minoh.osaka-u.ac.jp/flc/rus/\\
FORVO\\
https://ja.forvo.com/
\item{アプリ}
  \begin{description}
  \item[Anki] 暗記カード作成
  \item[Duolingo] 語学学習(ロシア語は英語話者向けのみ)
  \item[Memrise] 同上(日本語話者向けのロシア語講座あり)
\end{description}
\end{itemize}
\selectlanguage{russian}
\subsubsection*{今週の単語}
\begin{table}[ht]
  \begin{tabular}{ll}
    b\'ela{j1} vor\'ona & 風変わりな人\\
    vor\'ona & カラス \\
    enotov\'idna{j1} sob\'aka & タヌキ \\
    k\'arkat{p1} & (カラスがkar-karと)鳴く \\
    kot & 雄猫 \\
    k\'oxka & 雌猫 \\
    lev & ライオン \\
    lis\'a & キツネ \\
    oqk\'i & 眼鏡 \\
    perq\'atki & 手袋 \\
    poros{e0}nok & 子豚 \\
    svin{p1}\'\cyrya & 豚 \\
    hr\'\cyryu kat{p1} & (豚がhr{yu} hr{yu}と)啼く
  \end{tabular}
\end{table}
\clearpage
\subsubsection*{tyの用例 (A.チェーホフ \guillemotleft D{j1}d{j1} Van{j1}\guillemotright 第二幕 より 訳: 神西 清『ワーニャ伯父さん』)}
\begin{multicols}{2}
\begin{description}
\item[Elena Andreevna]
Sofi!
\item[Son{j1}]
Qto?
\item[Elena Andreevna]
Do kakih por vy budete dut{p1}s{j1} na men{j1}? Drug drugu my ne sdelali nikakogo zla. Zaqem {zh}e nam byt{p1} varagami? Polnote...
\item[Son{j1}]
{J1} sama hotela... \textit{(Obnimaet e{e0}.)} Dovol{p1}no serdit{p1}s{j1}.
\item[Elena Andreevna]
I otliqno.
\begin{center}
Obe vzvolnovany.
\end{center}
\item[Son{j1}]
P\'apa leg?
\item[Elena Andreevna]
Net, sidit v gostino{i0}... Ne govorim my drug s drugom po celym nedel{j1}m i, bog znaet, iz-za qego... \textit{(Uvidev, qto bufet otkryt.)} Qto {e1}to?
\item[Son{j1}]
Mihail L{p1}voviq u{zh}inal.
\item[Elena Andreevna]
I vino est{p1}... Dava{i0}te vyp{p1}em bruderxaft.
\item[Son{j1}]
Dava{i0}te.
\item[Elena Andreevna]
Iz odno{i0} r{yu}moqki... \textit{(Nalivaet.)} {E1}tak luqxe. Nu, znaqit \--- ty?
\item[Son{j1}]
Ty.
\begin{center}
P{p1}{yu}t i celu{yu}{t}s{j1}.
\end{center}
{J1} davno u{zh}e hotela mirit{p1}s{j1}, da vse kak-to sovestno bylo... \textit{(Plaqet.)}
\item[Elena Andreevna]
Qto {zh}e ty plaqex{p1}?
\item[Son{j1}]
Niqego, {e1}to {j1} tak.
\item[Elena Andreevna]
Nu, budet, budet... \textit{(Plaqet.)} Qudaqka, i {j1} zaplakala...
\end{description}

\columnbreak
\noindent
エレーナ ねえ、ソフィー。\\
ソーニャ なんですの?\\
エレーナ 一体いつまで、あなたはそんな顔をしているつもり? お互い、何ひとつ根に持つことなんかないじゃないの。どうして\ruby{敵}{かたき}同士にならなきゃいけないの? もう沢山だわ。……\\
ソーニャ わたしだって……(エレーナを抱きしめる)憤慨するのはもう沢山。\\
エレーナ それでなくちゃ\ruby{嘘}{うそ}よ。(二人とも感動のさま)\\
ソーニャ お父さま、おやすみになって?\\
エレーナ いいえ、客間で起きてらっしゃるの。……ほんとにこれで、もう何週間も口を\ruby{利}{き}かずにいたわねえ。べつにこれといって、わけもいわれもないのにさ……(食器棚のあいているのを見て)おや、どうしたの?\\
ソーニャ アーストロフさんが、お夜食をあがったの。\\
エレーナ \ruby{葡萄}{ぶどう}酒もあるわ。……仲直りのしるしに、ひとつ飲まない。\\
ソーニャ ええ、いいわ。\\
エレーナ このグラスで一緒にね。……(つぐ)そのほうがいいわ。じゃ、これでもう、ママと言ってくれるわね。\\
ソーニャ ええ。(飲んでキスする)わたし、ずっと前から仲直りがしたかったの。でも、なんだか恥ずかしくって……(泣く)\\
エレーナ おや、何で泣くの?\\
ソーニャ なんでもないの、ついわたし。\\
エレーナ さ、もういいわ、もういいわ……(泣く)おばかさんね、あたしまで、泣いちまったじゃないの。……
\end{multicols}
\end{document}
