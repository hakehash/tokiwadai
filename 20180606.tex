\documentclass[11pt]{jsarticle}
\usepackage{type1ec}
\usepackage[OT2,T1]{fontenc}
\usepackage[russian,english,japanese]{babel}
\usepackage{multicolpar}
\begin{document}
\subsection*{常盤台ロシア語学校(仮称) 単語テスト1}
\begin{flushright}
  2018-06-06
\end{flushright}
\selectlanguage{russian}
\subsubsection*{単語テスト 各2点\ (単語にはウダレーニエ(アクセントの記号)をふること)}
\begin{multicolpar}{2}
\--- \underline{\phantom{Что}} \underline{\phantom{это}}.\\
\--- \underline{\phantom{Это}} \underline{\phantom{ручка}}.

\noindent
\--- これは何ですか?\\
\--- これはペンです.
\end{multicolpar}
\begin{multicolpar}{2}
\--- Kto\ \underline{\phantom{вы}} ?\\
\--- \underline{\phantom{Я}} aspir\'antka.\\
\--- \underline{\phantom{А}} kto on?\\
\--- On \underline{\phantom{т\'ожe}} aspir\'ant.\\

\noindent
\--- あなたは誰ですか?\\
\--- 私は大学院生です.\\
\--- では彼は?\\
\--- 彼もまた, 大学院生です.
\end{multicolpar}
\begin{multicolpar}{2}
Ona sl\'u{zh}awa{j1}, a on rab\'oqi{i0}.

彼女は\underline{\phantom{勤 め 人}}で, 彼は\underline{\phantom{労 働 者}}です.
\end{multicolpar}
\begin{multicolpar}{2}
Mo\'\cyrya\ mat{p1}\ \---\ \underline{\phantom{домохоз\'яйка}}, a mo{i0} otec\ \---\ pension\'er.

私の母は主婦で, 私の父は年金生活者です.
\end{multicolpar}
\subsubsection*{今週の単語}
\begin{table}[ht]
  \begin{tabular}{ll}
    ban\'an & バナナ\\
    bar\'an & 雄羊\\
    d\'afni{j1} & ミジンコ\\
    DNK & デオキシリボ核酸\\
    ki\'osk & (街角や駅にある)売店\\
    kiosk{e0}r & 売店の売り子\\
    spirog\'ira & アオミドロ
  \end{tabular}
\end{table}
\subsubsection*{先週のフレーズ}
\noindent
\--- Perevod\'ite s r\'usskogo {j1}zyk\'a na {j1}p\'onski{i0}.\\
\--- Poqem\'u?\\
\\
\--- ロシア語から日本語に訳してください.\\
\--- どうして?\\
\\
単語: perevod\'it{p1} 翻訳する [不完], poqem\'u なぜ(疑問詞), poqem\'uqka なぜ?が口癖のように質問をする人
\end{document}
